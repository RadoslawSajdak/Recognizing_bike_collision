\chapter{Wstęp}
\label{cha:wstep}

Obecny rozwój mikroprocesorów, pozwala na tworzenie coraz bardziej złożonych urządzeń. Rozwój układów o niskim zużyciu energii, popycha elektronikę w kierunku małych, wielofunkcyjnych urządzeń. Połączenie tych dwóch procesów pozwala na stworzenie elastycznych urządzeń, których zastosowanie może zmieniać się jedynie dzięki oprogramowaniu.

%---------------------------------------------------------------------------

\section{Cele pracy}
\label{sec:celePracy}

Rowerzyści górscy, podczas samotnych wypraw rowerowych, wielokrotnie zastanawiają się, co w sytuacji, gdy ulegną wypadkowi podczas samotnej wyprawy?
Jak długo nikt nie wezwie pomocy? W ten sposób, powstał pomysł stworzenia urządzenia asystującego.
\newline
Celem pracy było stworzenie innowacyjnego urządzenia wykrywającego upadek na rowerze. Urządzenie miało być przymocowane do ramy roweru. Urządzenie, informację o wypadku wysyła przy użyciu modemu LTE-M opartego o interfejs szeregowy UART. Za lokalizowanie urządzenia, odpowiadać będzie moduł GPS oparty o magistralę $I^{2}C$. Całość, sterowana będzie przy użyciu mikrokontrolera.

\section{Analiza wymagań technicznych i dobór komponentów}

Docelowo, urządzenie miało zwiększać bezpieczeństwo podczas wypraw rowerowych. Musiało więc być bardzo energooszczędne. Minimalne wymaganie, to 24 godziny pracy na jednym ładowaniu baterii. Jednocześnie, nie może być zbyt duże, aby w łatwy sposób można było zamontować je na rowerze. Pobierana lokalizacja, miała mieć dokładność około 100m. Jest to dokładność wystarczająca, aby zobaczyć ranną osobę, leżący rower, lub usłyszeć wołanie o pomoc. Ważnym było, aby urządzenie, było w pełni niezależne od innych układów, jak np. telefon.
\newline
Planując pracę, zdecydowano się wykorzystać trzy moduły:
\begin{itemize}
    \item Akcelerometr
    \item GPS
    \item LTE
\end{itemize}
Dodatkowo, ze względu na łatwy w użyciu stos Bluetooth, wykorzystano również Bluetooth Low Energy, celem stworzenia bezprzewodowego interfejsu sterowania urządzeniem.
\newline
Ponieważna rynku dostępnych było wiele różnych modułów, poniżej dokonano ich porównania oraz wyboru układów, najbardziej pasujących do stworzonego rozwiązania.

\subsection{Akcelerometr}
Akcelerometry to układy, mierzące przyspieszenie. Mogą dokonać pomiaru przyspieszenia statycznego (np. Ziemskiego), lub dynamicznego, działającego z sił, działających na układ. W przypadku dostępnych na rynku akcelerometrów należy pamiętać, że w stanie spoczynku wskazują one przyspieszenie około $9.81\frac{m}{s^{2}}$. Można więc było, wykorzystać ten fakt, do implementacji algorytmu.
\newline
Obecnie, większość układów to układy integrujące akcelerometr i żyroskop w jednym układzie scalonym. Coraz częściej, można też spotkać magnetometr. Dla obecnych na rynku układów, wyróżniamy dwa najważniejsze parametry:
\begin{itemize}
    \item Zakres pracy akcelerometru - określany jako $\pm X_{g}$, a więc przyspieszenie w trzech kierunkach, podane jako wielokrotność przyspieszenia Ziemskiego. Zazwyczaj, wartość ta, mieści się w przedziale od kilku, do kilkunastu g.
    \item Zakres pomiarów żyroskopu - określony jako \emph{dps (degrees per second)}. Jeśli prędkość kątowa będzie większa, niż wybrany zakres, układ ulegnie nasyceniu
\end{itemize}
Głównym wymaganiem dotyczącym akcelerometru, była jego energooszczędność. Był to jedyny układ, który działa przez cały czas. Z tego powodu, akcelerometr powinien był nie tylko zużywać mało prądu, ale również posiadać różne tryby pracy. Dodatkowym atutem, była wbudowana pamięć, pozwalająca na buforowanie danych.
\newline
Spośród dostępnych na rynku układów, wybrane zostały trzy, dostępne w trakcie tworzenia pracy.

\subsubsection{MPU-6050}
Wybrany układ, jest 3-osiowym akcelerometem i żyroskopem. Korzysta on z magistrali $I^{2}C$. Zgodnie z dokumentacją układu, w normalnym trybie pracy, można spodziewać się ok. 3.8mA prądu, pobieranego przez układ.\cite{MPU6050} Wartość tę, można zredukować do nawet 10$\mu$A, ograniczając częstotliwość próbkowania akcelerometru do 1.25Hz i wyłączając żyroskop. Ponadto, układ posiada tryb niskiego zużycia energii, pozwalający uśpić nieaktywny układ. Sam akcelerometr, pracuje w zakresie $\pm$2g, $\pm$4g, $\pm$8g oraz $\pm$16g. Dodatkowo, układ posiada tzw. Digital Motion Pocessor (DMP), czyli układ wspomagający przetwarzanie danych w kierunku wykrywania gestów. Wbudowane FIFO, pozwala na buforowanie danych. Zaletą akcelerometru, są programowalne przerwania oraz przerwanie "High-G", wyzwalane w momencie przekroczenia zdefiniowanego przyspieszenia.

\subsubsection{LSM9DS1}
Układ ten, nie różni się znacząco od MPU-6050. Zgodnie z dokumentacją, jest on dodatkowo wyposażony w magnetometr. Największą z różnic, jest pobierany przez niego prąd. W przypadku LSM9DS1, akcelerometr w trybie normalnym, pobiera około 600$\mu$A.\cite{LSM9DS1} Niestety, wykorzystanie żyroskopu, dodaje kolejne 4mA. Żyroskop, posiada tryb niskiego zużycia energii, pozwalający ograniczyć zużycie energii do 1.9mA. Tym samym, układ nie jest w stanie zejść poniżej 1.96mA, co znacząco przekraczało domniemany pobór prądu.

\subsubsection{LSM6DSOX}
Ostatni z wybranych układów, był układem producenta ST. Jest on dedykowany do rozwiązań, o niskim zużyciu energii. Według dokumentacji, jego zużycie energii to 550$\mu$A.\cite{LSM6DSOX} Wartość ta, jest kilkukrotnie niższa, niż w przypadku MPU-6050. Co więcej, układ posiada tryb LowPower, w którym zużycie energii można ograniczyć do nawet 4$\mu$A. LSM6DSOX, posiada również 9kB FIFO, po którego napełnieniu, wystawiane jest przerwanie. Dodatkową zaletą, jest szesnaście programowalnych maszyn stanów. Pozwalają one na maksymalne ograniczanie zużycia energii, dzięki możliwości wyłączenia mikrokontrolera w trybie analizy danych. LSM6DSOX, do komunikacji wykorzystuje $I^{2}C$. Pozostałe jego parametry, są zbliżone do opisywanych wcześniej.

\subsubsection{Wybór akcelerometru}
Spośród trzech dostępnych układów, wybrany został LSM6DSOX. Jak pokazuje tabela \ref{tab:accelerometer}, układy te, są stosunkowo podobne. Decydującym elementem, okazały się być maszyny stanów, wbudowane w akcelerometr. Pozwoliły one znacząco ograniczyć zużycie energii całego układu oraz przyspieszyć proces tworzenia aplikacji.

\begin{table}[h]
\centering
\begin{tabular}{|c | c | c | c|}
    \hline
     & MPU-6050 & LSM9DS1 & LSM6DSOX \\
    \hline
    FIFO & 1kB  &   128B  & 9kB \\
    \hline
    Prąd pracy  & 3.8 mA & 600 $\mu$ & 550 $\mu$ \\
    \hline
    Low Power & 10-110 $\mu$A & 1.9-3.1 mA & 4-20 $\mu$A\\
    \hline
    Zakres pracy & 2-16g & 2-16g & 2-16g\\
    \hline
\end{tabular}
\caption{Porównanie dostępnych akcelerometrów}
\label{tab:accelerometer}
\end{table}

\subsection{Mikrokontroler}
Mikrokontroler, jest mózgiem układu. z tego powodu, wybrany układ musiał być bardzo wydajny, ale Jednocześnie energooszczędny. Kolejnym z wymagań, była obsługa interfejsu I$^{2}$C oraj UART. Ponieważ wybrany akcelerometr posiada dwa piny przerwań zewnętrznych, mikrokontroler musiał być gotowy je obsłużyć.

\subsubsection{ATmega328p}
Mikrokontroler ATmega328p, to jeden z najpopularniejszych układów na rynku. Jego niska cena i łatwość użycia, pozwala budować najróżniejsze aplikacje. Układ wspiera zewnętrzny zegar o częstotliwości do 16MHz. Posiada on UART, I$^{2}$C oraz SPI, co pozwoliło rozważyć go w pracy. Dodatkowo, w dokumentacji, można znaleźć informację o dwóch przerwaniach zewnętrznych, obsługiwanych przez układ.\cite{ATMEGA328P} Dla najwyższej częstotliwości, podczas pracy układ pobiera 9-14mA. Wartość ta, może zostać ograniczona do 2.8mA w trybie czuwania. W trybie zupełnego uśpienia, mikrokontroler pobiera pomiędzy 44, a 66$\mu$A.

\subsection{STM32F303K8}
Układ STM32F303K8 oparty jest na rdzeniu ARM Cortex M4. Posiada on wbudowany układ czasu rzeczywistego z funkcją wybudzania z kalendarzem. 11 wbudowanych liczników, pozwala na budowanie wyjątkowo złożonych aplikacji. Mikrokontroler wyposażono w SPI, I$^{2}$C, SPI, oraz 3 UARTy. W przypadku rdzeni Cortex, każdy z pinów może być skonfigurowany jako przerwanie, co daje nam aż 60 przerwań. \cite{STM32F303K8} Układ, może być taktowany zewnętrznym zegarem o częstotliwości do 32MHz, a więc dwukrotnie większym, niż w przypadku kontrolera ATmega. STM32F303K8 pobiera (w zależności od konfiguracji) 12.9 - 34mA w trybie pracy oraz 0.93-18.57$\mu$A w trybie głębokiego uśpienia. Warto zaznaczyć, że wartości maksymalneukład osiąga, gdy wszystkie peryferia są uruchomione. W przypadku tej pracy, dążono do minimalizacji ilości działających podzespołów.

\subsection{nRF528401}

\section{Podobne rozwiązania obecne na rynku}
Podobne rozwiązanie, zaproponowała firma Specialized. Inżynierowie stworzyli urządzenie o nazwie ANGI. Jest to mały układ, wyposażony w akcelerometr i żyroskop. Komunikuje się on przy użyciu Bluetooth z ich autorską aplikacją na telefon z iOS lub Androidem. Urządzenie, zamontowane na kask, wysyła powiadomienie do aplikacji, gdy rowerzysta uderza kaskiem w przeszkodę. Rozwiązanie to, ma szereg zalet, takich jak prostota budowy i bardzo niskie zużycie energii. Jest ono jednak uzależnione od telefonu, który w przypadku wycieczek górskich, czasem wygodniej zostawić w domu.

%---------------------------------------------------------------------------














